\documentclass{expose} % for english version
% \documentclass[german]{expose} % for german version

% general; update to concrete title and author
\title{Expose Dissertation Schuth}
\author{Rebecca Helene Schuth}
\date{\today}
% uncomment the suitable thesis name
%\thesistype{bachelor thesis}
%\thesistype{master thesis}
%\thesistype{Bachelorarbeit}
%\thesistype{Masterarbeit}
\thesistype{Dissertation}

\advisors{Prof. Jean Pierre Bergmann}


\addbibresource{references.bib}
\begin{document}
% create the title
\maketitle


\section{Problemstellung}
Ist es möglich, den Kristallisationsgrad eines Formteils im Spritzgießprozess so (lokal) zu regeln, dass sich die mechanischen Eigenschaften gezielt einstellen lassen. 

\section{Aktueller Forschungsstand}
Mit Hilfe der Avrami-Gleichung lässt sich die eine Gefügeänderung, also Kristallisation, bei isothermen Bedigungen beschreiben (1930er).
Nakamura-Ziabicki Modell beschreibt die nicht-iothermische Kristallisationskinetik (1970er).  

\\
Avrami Gleichung und Nakamura Gleichung.  
\\

\section{Fragestellung der Arbeit}
Es gilt zu klären, wie sich die Abkühlrate auf die sich ausprägende Kristallisationsgrad des Formteils auswirkt. 
Hierzu wird der Spritzgießprozess in seiner Gesamtheit betrachtet (Parameter der Plastifiziereinheit, Werkzeugparameter).
Weiterhin wird erforscht, welche mechanischen Eigenschaften am geeigntsten sind, um sie im Prozess zu regeln (Stichwort Zykluszeit).


\section{Erkenntnisse des Verfassers}
\section{Ziel}
\section{Theorien auf die sich bezogen wird}
\section{Methoden des Vorgehens}
unterschiedliche Materialien untersuchen,  

\section{Verwendete Quellen / Material}
\section{Vorläufige Gliederung}
\section{Zeitplan}


%
% \section{Introduction}
%  (max 1 page)
% \begin{itemize}
%     \item general problem space
%     \item e.g. using an example for short introduction
%     \item which concrete topic is in focus
% \end{itemize}

% \section{Motivation and Goals}
%  (max 0.75 page)
% \begin{itemize}
%     \item describe topic and goals in detail
%     \item what are possible sub-problems
% \end{itemize}

% \section{First Ideas and possbile Approaches}
%  (max 0.75 page)

% \begin{itemize}
%     \item describe some ideas for solving the problem and
%     \item explain some general approaches to solve the thesis
% \end{itemize}

% \section{Timeplan}
% \begin{itemize}
%     \item a short timeplan for the thesis
%     \item define some 'milestones'
% \end{itemize}

% \section{Example Template Usage}
% You can cite something with~\cite{Berlin}.
% Or \citeauthor{raake2014quality} for Authors~\cite{raake2014quality}.

% Include Equations using:
% \[ e = m \cdot c^2 \]

% Or inline $x=42$.

% \begin{figure}[tbh!]
% \centering
% %\includegraphics[width=\textwidth]{path-to-figure}
% \missingfigure{missing figure}
% \caption{Caption of a figure, }
% \label{pic::example1}
% \end{figure}
% Figures can be included, see Fig~\ref{pic::example1}.


% \begin{table}[h]
% \centering
% \caption{Caption of a Table}
% \label{tbl::example1}
% \begin{tabular}{lr}
% \toprule
% \textbf{XY}  & \textbf{XY} \\
% \midrule
% 1   & 2 \\
% \midrule
% 3 & 4 \\
% \bottomrule
% \end{tabular}
% \end{table}

% For tables see Tables~\ref{tbl::example1}.

% Todo markers: \todo{here}
% or \todoI{inline}

% \R{red text}

% \note{small note}

\printbibliography

\clearpage

% remove in final version
\listoftodos{Todos}

\end{document}
